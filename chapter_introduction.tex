\chapter{Introduction}
\label{chap_introductiont}

Complex nonlinear dynamics are common to many systems, including prominent examples in financial systems, climate, ecology \cite{May_2008, Scheffer_2009}. Because of the way in which endogenous processes, external drivers, and stochasticity interact to produce these complex dynamics, it can be challenging to apply the traditional approach of mathematical models. In many biological models, for example, system behavior can be highly sensitive to model structure and parameters, making it difficult to choose the correct set of equations \cite{Wood_1999}. Even when the true system equations are known, correctly fitting them to data in the presence of nonlinear interactions and observation error is not guaranteed \cite{Perretti_2013}. In such cases, it can be tempting to take a statistical approach and treat such systems as stochastic; however applying a purely statistical framework ignores the potential for nonlinear effects to generate extreme events, leading to such statistical improbabilities as 25-sigma events \cite{Dowd_2011}. 

\section{The Dangers of Using Incorrect Models}

George Box famously said that "all models are wrong, some are useful" \cite{Box_1987}. The implications of this statement are that models are simplified approximations of reality, but can give meaningful insights or predictions when used appropriately. Furthermore, an oft-ignored facet of modeling is that different models have different intended uses \cite{Peters_1991}. For example, some models are designed primarily to understand mechanisms and concepts or to evaluate hypotheses, but not to be used for prediction. End-to-end ecosystem models embody this philosophy by including the major relevant processes in order to understand how they interact and to determine the sensitivity of system behavior to changes to these processes \cite{Fulton_2010}. Such models are not expected to be used directly for prediction or management, because of how uncertainty in model structure affects predictions \cite{Kaplan_2013}. Correct application of ecological models therefore requires caution.

Although we acknowledge the reality of nonlinear systems, the linear framework remains widely used. For example, a classic meta-analysis by \cite{Myers_1998} examined 74 statistically significant and published correlations between fish recruitment and the environment. Of these 74 correlations, only 28 held up when re-examined with newer data, suggesting that the remaining 46 results were false positives. However, given the prevalence of nonlinearity in marine ecosystems \cite{Hsieh_2005, Glaser_2014}, it is much more likely that these alleged false positives are actually ``mirage correlations'', a phenomenon in nonlinear systems where coupled variables can appear linearly correlated, but only over certain time segments \cite{Sugihara_2012}. Instead of discouraging scientists from publishing environment-recruitment relationships, these findings suggest that there are actually many more true associations that remain unpublished because they do not pass the threshold for linear significance. This mismatch between the known reality of the world and published research indicates a need for more sophisticated tools to continue scientific progress.

\section{Empirical Dynamic Modeling}

As an alternative to the classical approach of assuming fixed equations, the framework of Empirical Dynamic Modeling (EDM) seeks to empirically reconstruct the behavior of a system solely from the data \cite{Sugihara_1990, Sugihara_1994, Dixon_1999, Hsieh_2005, Sugihara_2012, Deyle_2013, Ye_2015}. Whereas a traditional mathematical model uses equations to describe behavior and the interactions between variables, the EDM approach instead extracts this information from the time-dependent relationships between variables and their lags. Here, the key is that time series do not represent random observations, but are actually ordered recordings of the system behavior as viewed from the perspective of the corresponding variable. 

The benefits to this nonparametric and equation-free approach are numerous. Because no equations are assumed, the model's depiction of behavior can flexibly accommodate whatever is observed in the data. Moreover, as demonstrated in chapter \ref{chap_multiembed}, the associations between multiple time series from the same system also contain information that can be exploited to produce better models. However, there are limitations to EDM. Because EDM is based on recovering dynamics from data, it can only describe what has been observed in the data, and is therefore constrained by data quality and quantity. In contrast, the traditional equation-based approach can use of assumed parameters or equations when data are insufficient.

Nevertheless, EDM remains a powerful framework for studying dynamic systems. In this dissertation, EDM is applied to  time series data of sockeye salmon populations from the Fraser River in British Columbia, Canada to understand both the dynamics of salmon recruitment, and to produce forecast models that can be compared with traditional fisheries models. In addition, new methods within the EDM framework are developed to expand its capabilities, and which yield new insights into leveraging information in complex systems.

\section{Summary of Chapters}

Chapter \ref{chap_salmon_environment} lays the basic groundwork of analysis for sockeye salmon populations of the Fraser River, identifying predictable recruitment dynamics using low-dimensional embeddings, and detecting nonlinear (i.e., state-dependent) interactions between juveniles and the environment. Multivariate EDM models (sensu \cite{Dixon_1999, Deyle_2013}) are constructed that incorporate predictive information contained within environmental proxies. By outperforming traditional fisheries models that assume that stock size and the environment act independently, the utility of the equation-free EDM approach for recovering nonlinear interactions is demonstrated.

Chapter \ref{chap_salmon_regimes} investigates the question of whether apparent changes in mean salmon productivity reflect true ``regime shifts'' (changes to system dynamics and behavior) or represent artifacts arising from nonlinear behavior. Whereas classical stock-recruitment models are known to fit the data better when it is partitioned into climate regimes of the North Pacific Ocean \cite{Beamish_2004a}, multivariate EDM models show no such preference, with equivalent forecast skill regardless of how the data are partitioned. These results suggest that an appropriately nonlinear perspective, such as the flexible EDM models used here and in chapter \ref{chap_salmon_environment}, does not support the notion of regime-like behavior in this system.

Chapter \ref{chap_ccm_time_delays} addresses several technical issues when applying the method of convergent cross mapping (CCM) to identify causal interactions. When introducing CCM, \cite{Sugihara_2012} noted that it can be difficult to distinguish between cases where two variables exhibit bidirectional causality (i.e., effects in both directions as in a feedback loop) and cases where one variable so strongly affects the other that the two variables become entrained (i.e., ``generalized synchrony'', as in \cite{Rulkov_1995}). By examining the optimal time delay associated with CCM, however, these two cases can be distinguished, because of how causes must precede effects. Moreover, this approach is also shown to discriminate between direct and indirect causal effects (created by the transitivity of causal chains), and to more precisely identify the strength of causal effects.

Chapter \ref{chap_multiembed} develops a multimodel approach within the EDM framework called Multiview Embedding (MVE). MVE is based on the corollary of Takens' Theorem \cite{Takens_1981} that information about dynamic behavior is contained within all time series variables of a system. As such, different reconstructions of the dynamics can be combined to give more precise models because of duplicated information. Importantly, EDM is uniquely positioned to exploit this opportunity, because it does not use fixed equations to represent the dynamics and is therefore agnostic to the transformation of system equations that would otherwise be required for a parametric approach. MVE is tested on several model systems and a mesocosm experiment, demonstrating significantly better forecasts from time series as short as 25 points.

Appendix \ref{chap_redm_package} is the user guide for the rEDM software package. rEDM collects the various methods for EDM along with several example datasets into a single package for the R statistical language. The user guide is divide into three main sections: (1) instructions for downloading and installing the package; (2) a basic overview of EDM concepts and package functions; and (3) two examples of applying EDM to time series data from real systems. 